\small

Häufig müssen Ähnlichkeiten von Bäumen, die aus dem gleichen Satz Blattknoten bestehen, jedoch unterschiedliche Topologien aufweisen, quantitativ verglichen werden. Obwohl diese Problematik bekannt ist und viele Metriken vorhanden sind, unterliegen diese verschiedenen Einschränkungen und sind nicht für größere Bäumen skalierbar. Diese Arbeit stellt die Basis einer neuartigen informationstheoretischen Metrik für Bäume, genannt Baum-Transinformation dar, die durch die von den betrachteten Bäumen geteilte Information ausgewertet werden kann. Die Metrik basiert auf der besten übereinstimmenden Teilung der von zwei Bäumen erzeugten Blattknoten eines Satzes. Diese Methode kann verwendet werden, um die Qualität des hierarchischen Clusterns zu bewerten und die Ergebnisse zu interpretieren.

Zusätzlich zu dieser neuartigen Metrik wird in dieser Arbeit eine neue Technik zur Bewertung der angepassten Transinformation, basierend auf paarweisen Vertauschungen, vorgestellt. Diese Berechnungsmethode ist wesentlich schneller und kann daher für den Vergleich größerer Bäume verwendet werden.

Alle Experimente wurden sowohl mit synthetischen als auch mit realen Datensätzen durchgeführt, um die Effizienz des Ansatzes in verschiedenen Situationen zu veranschaulichen. Die vorgeschlagene Metrik übertrifft bereits bestehende Metriken sowohl in der Qualität als auch in der Laufzeit.

Schlüsselwörter: Hierarchisches Clustering, Dendrogram, Bäume, angepasste Transinformation, Informationstheorie.
